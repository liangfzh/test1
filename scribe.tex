\documentclass[11pt]{article}

\usepackage{mathspec, indentfirst}
\setmainfont{TeX Gyre Pagella}
\setsansfont{TeX Gyre Heros}
\setmonofont{TeX Gyre Cursor}
\setmathfont(Digits,Latin,Greek){TeX Gyre Pagella}

% Set 1 inch margin on each side.  Page numbers, if any, appear in the margin.
\setlength{\hoffset}{0in}
\setlength{\oddsidemargin}{0in}
%\setlength{\evensidemargin}{0in}
\setlength{\textwidth}{\paperwidth}
\addtolength{\textwidth}{-1in}
\addtolength{\textwidth}{-\hoffset}
\addtolength{\textwidth}{-\oddsidemargin}
\addtolength{\textwidth}{-1in} % Set the right margin here
\setlength{\voffset}{0in} 
\setlength{\topmargin}{0in}
\setlength{\headheight}{0in}
\setlength{\headsep}{0in}
\setlength{\textheight}{\paperheight}
\addtolength{\textheight}{-1in}
\addtolength{\textheight}{-\topmargin}
\addtolength{\textheight}{-\headheight}
\addtolength{\textheight}{-\headsep}
\addtolength{\textheight}{-1in} % Set the bottom margin here

\newcommand{\handout}[5]{
  \noindent
  \begin{center}
  \framebox{
    \vbox{
      \hbox to 5.78in { {\bf Computer Security } \hfill #2 }
      \vspace{4mm}
      \hbox to 5.78in { {\Large \hfill #5  \hfill} }
      \vspace{2mm}
      \hbox to 5.78in { {\em #3 \hfill #4} }
    }
  }
  \end{center}
  \vspace*{4mm}
}

\newcommand{\lecture}[4]{\handout{#1}{#2}{#3}{Scribe: #4}{Lecture #1}}

\begin{document}

\lecture{NUMBER --- DATE, 2015}{Spring 2015}{Prof.\ Hao Chen}{YOUR NAME}

\section{Overview}

In the last lecture we \ldots.

In this lecture we \ldots.

\section{Main Section}

We begin by describing the problem \ldots.
Make sure to use sections and subsections.

\subsection{Blah blah blah}
Here is a subsection.

\subsubsection{Blah blah blah}
Here is a subsubsection. You can use these as well.

\paragraph{Question:}
How would you use boldface?

\paragraph{Example:}
This is an example showing how to use boldface to 
help organize your lectures.


\paragraph{Some Formatting.}
Here is some formatting that you can use in your notes:
\begin{itemize}
\item {\em Item One} -- This is the first item.
\item {\em Item Two} -- This is the second item.
\item \dots and here are other items.
\end{itemize}

If you need to number things, you can use this style:
\begin{enumerate}
\item {\em Item One} -- Again, this is the first item.
\item {\em Item Two} -- Again, this is the second item.
\item \dots and here are other items.
\end{enumerate}

\end{document}
